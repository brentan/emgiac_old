%%%%%%%%%%%%%%%%%%%%%%%%%%%%%%%
%% giac/hevea interaction code definitions
%%%%%%%%%%%%%%%%%%%%%%%%%%%%%%%
%% Type one of the commands \loadgiacjs or \loadgiacjsonline 
%% to load the javascript code (from hard disk or Internet)
%% Commands \giacinput{} or \giacinputmath{} or \giacinputbigmath{} 
%% \giaccmd{} or \giaccmdmath{} or \giaccmdbigmath{} 
%% or \begin{giacprog}...\end{giacprog}
\usepackage{hevea} 
\usepackage{listings}
\usepackage{fancyvrb}
\newcommand{\loadgiacjs}{
\ifhevea
\@print{<script language="javascript"> 
var Module = { preRun: [], postRun: []};
</script>
<script src="file:///usr/share/giac/doc/giac.js" async></script> 
<script language="javascript"> 
 var UI = {
  histcount:0,
  usemathjax:false,
  eval_form: function(field){
    caseval(field.name.value+':='+field.valname.value);
    console.log(field.prog.value);
    var s=caseval(field.prog.value);
    var is_svg=s.substr(1,4)=='<svg';
    if (is_svg) field.result.innerHTML=s.substr(1,s.length-2);
    else field.result.innerHTML=s;
  },
  latexeval:function(text){
    var tmp=text;
     if (UI.usemathjax){
       tmp=caseval('latex(quote('+tmp+'))');
       tmp='$$'+tmp.substr(1,tmp.length-2)+'$$';
       return tmp;
     }
     tmp=caseval('mathml(quote('+text+',1))');
     tmp=tmp.substr(1,tmp.length-2);
    return tmp;   
  },
  ckenter:function(event,field,mode){
    var key = event.keyCode;
    if (key != 13 || event.shiftKey) return true;
    var tmp=field.value;
    tmp=caseval(tmp);
    if (mode==1){
      if (tmp.charCodeAt(0)==34) tmp=tmp.substr(1,tmp.length-2); 
   }
   if (mode==2){
     tmp=UI.latexeval(text);
   }
   field.nextSibling.nextSibling.innerHTML='&nbsp;'+tmp;
   if (UI.usemathjax) MathJax.Hub.Queue(["Typeset",MathJax.Hub,field.nextSibling.nextSibling]);
   if (event.preventDefault) event.preventDefault();
    return false;
  },
   exec: function(field){
     if (field.nodeName=="BUTTON"){
        field.click();
        return;
     }
     var f=field.firstChild;
     while (f){
       UI.exec(f);
       f=f.nextSibling;
     }
   }
 };
 window.onload = function(e){
  var ua = window.navigator.userAgent;
  var old_ie = ua.indexOf('MSIE ');
  var new_ie = ua.indexOf('Trident/');
  if ((old_ie > -1) || (new_ie > -1) || window.chrome){
     UI.usemathjax=true;
     alert("Your browser does not support MathML, using MathJax for 2d rendering. Consider switching to Firefox for better rendering and faster results");
  }
  if (UI.usemathjax){
    var script = document.createElement("script");
    script.type = "text/javascript";
    script.src  = "mathjax/MathJax.js?config=TeX-AMS-MML_HTMLorMML";
    document.getElementsByTagName("head")[0].appendChild(script);
  }
  caseval=Module.cwrap('_ZN4giac7casevalEPKc',  'string', ['string']);
  caseval('set_language(1)');
  if (confirm('Run history commands?')){
    UI.exec(document.documentElement);
  }
 };
</script>
}
\else
The HTML version of this text has interactive fields running Giac
computer algebra system commands.
\fi
}
\newcommand{\loadgiacjsonline}{
\ifhevea
\@print{<script language="javascript"> 
var Module = { preRun: [], postRun: []};
</script>
<script src="http://www-fourier.ujf-grenoble.fr/~parisse/giac.js" async></script> 
<script type="text/javascript"
  src="http://cdn.mathjax.org/mathjax/latest/MathJax.js?config=TeX-AMS-MML_HTMLorMML">
</script>
<script language="javascript">
 var UI = {
  histcount:0,
  usemathjax:false,
  eval_form: function(field){
    caseval(field.name.value+':='+field.valname.value);
    console.log(field.prog.value);
    var s=caseval(field.prog.value);
    var is_svg=s.substr(1,4)=='<svg';
    if (is_svg) field.result.innerHTML=s.substr(1,s.length-2);
    else field.result.innerHTML=s;
  },
  latexeval:function(text){
    var tmp=text;
     if (UI.usemathjax){
       tmp=caseval('latex(quote('+tmp+'))');
       tmp='$$'+tmp.substr(1,tmp.length-2)+'$$';
       return tmp;
     }
     tmp=caseval('mathml(quote('+tmp+',1))');
     tmp=tmp.substr(1,tmp.length-2);
    return tmp;   
  },
  ckenter:function(event,field,mode){
    var key = event.keyCode;
    if (key != 13 || event.shiftKey) return true;
    var tmp=field.value;
    tmp=caseval(tmp);
    if (mode==1){
      if (tmp.charCodeAt(0)==34) tmp=tmp.substr(1,tmp.length-2); 
   }
   if (mode==2){
     tmp=UI.latexeval(tmp);
   }
   field.nextSibling.nextSibling.innerHTML=tmp;
   if (UI.usemathjax) MathJax.Hub.Queue(["Typeset",MathJax.Hub,field.nextSibling.nextSibling]);
   if (event.preventDefault) event.preventDefault();
    return false;
  },
  exec: function(field){
     if (field.nodeName=="BUTTON"){
        field.click();
        return;
     }
     var f=field.firstChild;
     while (f){
       UI.exec(f);
       f=f.nextSibling;
     }
   }
 };
 window.onload = function(e){
  var ua = window.navigator.userAgent;
  var old_ie = ua.indexOf('MSIE ');
  var new_ie = ua.indexOf('Trident/');
  if ((old_ie > -1) || (new_ie > -1) || window.chrome){
     UI.usemathjax=true;
     alert("Your browser does not support MathML, using MathJax for 2d rendering. Consider switching to Firefox for better results");
  }
  if (UI.usemathjax){
    var script = document.createElement("script");
    script.type = "text/javascript";
    script.src  = "http://cdn.mathjax.org/mathjax/latest/MathJax.js?config=TeX-AMS-MML_HTMLorMML";
    document.getElementsByTagName("head")[0].appendChild(script);
  }
  caseval=Module.cwrap('_ZN4giac7casevalEPKc',  'string', ['string']);
  caseval('set_language(1)');
  if (confirm('Run history commands?')){
    UI.exec(document.documentElement);
  }
 };
</script>
}
\else
The HTML version of this text has interactive fields running Giac
computer algebra system commands.
\fi
}
\ifhevea
\newenvironment{giacprog}{
\verbatim}
{\endverbatim 
\@print{<button onclick="var field=parentNode.previousSibling; var tmp=field.innerHTML;if(tmp.length==0) tmp=field.value;var t=createElement('TEXTAREA');t.style.fontSize=16;t.cols=60;t.rows=10;t.value=tmp;tmp=caseval(tmp);if (tmp.charCodeAt(0)==34) tmp=tmp.substr(1,tmp.length-2);nextSibling.innerHTML=tmp; field.parentNode.insertBefore(t,field);field.parentNode.removeChild(field);">ok</button><span></span><br>
}
}
\else
\newenvironment{giacprog}
{
\VerbatimEnvironment
\begin{Verbatim}
}
{
\end{Verbatim}
}
\fi

% for example \giaccmd{factor}{x^4-1}
\newcommand{\giaccmd}[3][style="width:400px;height:20px;font-size:large"]{
\ifhevea
\@print{<textarea}
\@getprint{#1>#3}
\@print{</textarea><button onclick="var tmp=caseval(}
\@getprint{'#2('}
\@print{+previousSibling.value+')');if (tmp.charCodeAt(0)==34) tmp=tmp.substr(1,tmp.length-2); nextSibling.innerHTML='&nbsp;'+tmp">}
\@getprint{#2}
\@print{</button><span></span><br>}
\else
\lstinline@#2(#3)@
\fi
}
\newcommand{\giacinput}[2][style="width:400px;height:20px;font-size:large"]{
\ifhevea
\@print{<textarea onkeypress="UI.ckenter(event,this,1)"}
\@getprint{#1>#2}
\@print{</textarea><button onclick="var tmp=caseval(previousSibling.value);if (tmp.charCodeAt(0)==34) tmp=tmp.substr(1,tmp.length-2); nextSibling.innerHTML='&nbsp;'+tmp">ok</button><span></span><br>}
\else
\lstinline@#2@
\fi
}
\newcommand{\giaccmdmath}[3][style="width:400px;height:20px;font-size:large"]{\ifhevea
\begin{rawhtml}<br><textarea \end{rawhtml}
\@getprint{#1>#3}
\begin{rawhtml}</textarea><button onclick="var tmp=caseval(\end{rawhtml} 
\@getprint{'#2('+}
\begin{rawhtml}previousSibling.value+')');  tmp=UI.latexeval(tmp);nextSibling.innerHTML='&nbsp;'+tmp; if (UI.usemathjax) MathJax.Hub.Queue(['Typeset',MathJax.Hub,nextSibling]);
">\end{rawhtml}
\@getprint{#2}
\begin{rawhtml}</button><span></span><br>\end{rawhtml}
\else
\lstinline@#2(#3)@
\fi
}
\newcommand{\giacinputmath}[2][style="width:400px;height:20px;font-size:large"]{\ifhevea
\begin{rawhtml}<br><textarea onkeypress="UI.ckenter(event,this,2)"\end{rawhtml}
\@getprint{#1>#2} 
\begin{rawhtml}</textarea><button onclick="var tmp=caseval(previousSibling.value); tmp=UI.latexeval(tmp);nextSibling.innerHTML='&nbsp;'+tmp; if (UI.usemathjax) MathJax.Hub.Queue(['Typeset',MathJax.Hub,nextSibling])">ok</button><span></span><br>\end{rawhtml}
\else
\lstinline@#2@
\fi
}
\newcommand{\giaccmdbigmath}[3][style="width:800px;height:20px;font-size:large"]{\ifhevea
\begin{rawhtml}<br><textarea \end{rawhtml}
\@getprint{#1>#3} 
\begin{rawhtml}</textarea><button onclick="var tmp=caseval(\end{rawhtml} 
\@getprint{'#2('+}
\begin{rawhtml}previousSibling.value+')');  tmp=UI.latexeval(tmp);nextSibling.innerHTML=tmp; if (UI.usemathjax) MathJax.Hub.Queue(['Typeset',MathJax.Hub,nextSibling])">\end{rawhtml}
\@getprint{#2}
\begin{rawhtml}</button><div style="width:800px;max-height:200px;overflow:auto;color:blue;text-align:center"></div><br>\end{rawhtml}
\else
\lstinline@#2(#3)@
\fi
}
\newcommand{\giacinputbigmath}[2][style="width:800px;height:20px;font-size:large"]{\ifhevea
\begin{rawhtml}<div><textarea onkeypress="UI.ckenter(event,this,2)"\end{rawhtml}
\@getprint{#1>#2} 
\begin{rawhtml}</textarea><button onclick="var tmp=caseval(previousSibling.value); tmp=UI.latexeval(tmp);nextSibling.innerHTML=tmp; if (UI.usemathjax) MathJax.Hub.Queue(['Typeset',MathJax.Hub,nextSibling])">ok</button><div style="width:800px;max-height:200px;overflow:auto;color:blue;text-align:center"></div></div>\end{rawhtml}
\else
\lstinline@#2@
\fi
}
\newcommand{\giaclink}[2][Test online]{\ifhevea
\begin{rawhtml}<a href=\end{rawhtml}\@getprint{"#2"}
\begin{rawhtml} target="_blank">\end{rawhtml}
\@getprint{#1}
\begin{rawhtml}</a>\end{rawhtml}
\else
\fi
}
% \giacslider{name}{mini}{maxi}{step}{value}{prog}
\newcommand{\giacslider}[6]{
\ifhevea
\begin{rawhtml}
<form onsubmit="setTimeout(function(){UI.eval_form(form);});return false;">
<input type="text" name="name" size="1" value=
\end{rawhtml}
\@getprint{"#1">}
\begin{rawhtml}
=<input type="number" name="valname" onchange="UI.eval_form(form);" value=
\end{rawhtml}
\@getprint{"#5">}
\begin{rawhtml}
<input type="button" value="-" onclick="valname.value -= stepname.value;UI.eval_form(form);">
<input type="button" value="+" onclick="valname.value -= -stepname.value;UI.eval_form(form);">
<input type="number" name="stepname" value=
\end{rawhtml}
\@getprint{"#4">}
\begin{rawhtml}
<input type="range" name="rangename"
onclick="valname.value=value;UI.eval_form(form);" value=
\end{rawhtml}
\@getprint{"#5" min="#2" max="#3" step="#4">}
\begin{rawhtml}
<br>
<textarea name="prog" onchange="UI.eval_form(form)" style="width:400px;height:180px;vertical-align:bottom;font-size:large">
\end{rawhtml}
\@getprint{#6}
\begin{rawhtml}
</textarea>
<output name="result" style="vertical-align:bottom"></output>
</form>
\end{rawhtml}
\else
\fi
}
%%%%%%%%%%%%%%%%%%%%%%%%%%%%%%%
%% giac/hevea end code definitions
%%%%%%%%%%%%%%%%%%%%%%%%%%%%%%%
