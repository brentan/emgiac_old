% Pour compiler ce fichier, vous devez d'abord installer
% hevea: http://hevea.inria.fr
% giac.tex: http://www-fourier.ujf-grenoble.fr/~parisse/giac/giac.tex
% copier hevea.sty dans le repertoire courant
% ensuite vous pouvez tester avec la commande hevea test 
\documentclass[a4paper,11pt]{article}
%\textwidth 11,8 cm
%\textheight 17 cm
\textheight 23 cm
\usepackage{graphicx}
\usepackage{amsmath}
\usepackage{amsfonts}
\usepackage{amssymb}
\usepackage{stmaryrd}
\usepackage{makeidx}
\usepackage{times}
\usepackage[utf8]{inputenc}
\usepackage[T1]{fontenc}
\usepackage{latexsym}
\usepackage{graphicx}
%\usepackage{pst-plot}
\usepackage{ifpdf}
\ifpdf
 \usepackage[pdftex,colorlinks]{hyperref}
\else
 \usepackage[ps2pdf,
            breaklinks=true,
            colorlinks=true,
            linkcolor=red,
            citecolor=green
            ]{hyperref}
\fi

%%%%%%%%%%%%%%%%%%%%%%%%%%%%%%%
%% giac/hevea interaction code definitions
%%%%%%%%%%%%%%%%%%%%%%%%%%%%%%%
%% Type one of the commands \loadgiacjs or \loadgiacjsonline 
%% to load the javascript code (from hard disk or Internet)
%% Commands \giacinput{} or \giacinputmath{} or \giacinputbigmath{} 
%% \giaccmd{} or \giaccmdmath{} or \giaccmdbigmath{} 
%% or \begin{giacprog}...\end{giacprog}
%% contact hevea to output <frameset> </frameset> for canvas and menu?
\usepackage{hevea} 
\usepackage{listings}
\usepackage{fancyvrb}
\newcommand{\loadgiacjs}{
\ifhevea
\@print{
<button onclick="UI.exec(document.documentElement);" title='Click here to execute all commands'>Execute</button>
<canvas id='canvas' width=0 height=0   onmousedown="UI.canvas_pushed=true;UI.canvas_lastx=event.clientX; UI.canvas_lasty=event.clientY;"  onmouseup="UI.canvas_pushed=false;" onmousemove="UI.canvas_mousemove(event,'')"></canvas>
<textarea id="output" rows=2 cols="80" title="Console"></textarea>
<script language="javascript"> 
var Module = { preRun: [], postRun: [], 
        print: (function() {
          var element = document.getElementById('output');
          element.value = ''; // clear browser cache
          return function(text) {
            element.style.display='inherit';
            element.value += text + "\n";
            element.scrollTop = 99999; // focus on bottom
          };
        })(),
      canvas: document.getElementById('canvas'),};
</script>
<script src="file:///usr/share/giac/doc/giac.js" async></script> 
<script language="javascript"> 
 var UI = {
  histcount:0,
  usemathjax:false,
  canvas_pushed:false,
  canvas_lastx:0,
  canvas_lasty:0,
  canvas_mousemove:function(event,no){
    if (UI.canvas_pushed){
      // Module.print(event.clientX);
      if (UI.canvas_lastx!=event.clientX){
        if (event.clientX>UI.canvas_lastx)
          giac3d('r'+no);
        else
          giac3d('l'+no);
        UI.canvas_lastx=event.clientX;
      }
      if (UI.canvas_lasty!=event.clientY){
        if (event.clientY>UI.canvas_lasty)
          giac3d('d'+no);
        else
          giac3d('u'+no);
        UI.canvas_lasty=event.clientY;
      }
    }
  },  
  render_canvas:function(field){
   var n=field.id;
   if (n && n.length>5 && n.substr(0,5)=='gl3d_'){
    Module.print(n);
    var n3d=n.substr(5,n.length-5);
    giac3d(n3d);
    return;
   }
   var f=field.firstChild;
   for (;f;f=f.nextSibling){
     UI.render_canvas(f);
   }
  },
  caseval:function(text){
    var s="not evaled",err;
    try {
       s= giaceval(text);
    } catch (err) { s=err.message;}
    var is_3d=s.length>5 && s.substr(0,5)=='gl3d ';
    if (is_3d){
	var n3d=s.substr(5,s.length-5);
	s = '<canvas id="gl3d_'+n3d+'" onmousedown="UI.canvas_pushed=true;UI.canvas_lastx=event.clientX; UI.canvas_lasty=event.clientY;" onmouseup="UI.canvas_pushed=false;" onmousemove="UI.canvas_mousemove(event,'+n3d+')" width=400 height=250></canvas>';
    }
   //console.log(s);
    return s;
  },
  eval_form: function(field){
    giaceval('assume('+field.name.value+'='+field.valname.value+')');
    var s=UI.caseval(field.prog.value);
    var is_svg=s.substr(1,4)=='<svg';
    if (is_svg) field.parentNode.lastChild.innerHTML=s.substr(1,s.length-2);
    else field.parentNode.lastChild.innerHTML=s;
   UI.render_canvas(field.parentNode.lastChild);
  },
  latexeval:function(text){
    var tmp=text;
     if (UI.usemathjax){
       tmp=giaceval('latex(quote('+tmp+'))');
       tmp='$$'+tmp.substr(1,tmp.length-2)+'$$';
       return tmp;
     }
     tmp=giaceval('mathml(quote('+text+',1))');
     tmp=tmp.substr(1,tmp.length-2);
    return tmp;   
  },
  ckenter:function(event,field,mode){
    var key = event.keyCode;
    if (key != 13 || event.shiftKey) return true;
    var tmp=field.value;
    tmp=UI.caseval(tmp);
    if (mode==1){
      if (tmp.charCodeAt(0)==34) tmp=tmp.substr(1,tmp.length-2); 
   }
   if (mode==2){
     tmp=UI.latexeval(text);
   }
   field.nextSibling.nextSibling.innerHTML='&nbsp;'+tmp;
   UI.render_canvas(field.nextSibling.nextSibling);
   if (UI.usemathjax) MathJax.Hub.Queue(["Typeset",MathJax.Hub,field.nextSibling.nextSibling]);
   if (event.preventDefault) event.preventDefault();
    return false;
  },
   exec: function(field){
     if (field.nodeName=="BUTTON"){
        field.click();
        return;
     }
     if (field.nodeName=="FORM"){
        UI.eval_form(field);
        return;
     }
     var f=field.firstChild;
     while (f){
       UI.exec(f);
       f=f.nextSibling;
     }
   }
 };
 window.onload = function(e){
   var isFirefox = typeof InstallTrigger !== 'undefined';   // Firefox 1.0+
   var isSafari = Object.prototype.toString.call(window.HTMLElement).indexOf('Constructor') > 0;
  var ua = window.navigator.userAgent;
  var old_ie = ua.indexOf('MSIE ');
  var new_ie = ua.indexOf('Trident/');
  if ((!isFirefox && !isSafari) || (old_ie > -1) || (new_ie > -1) || window.chrome){
     UI.usemathjax=true;
     alert("Your browser does not support MathML, using MathJax for 2d rendering. Consider switching to Firefox for better rendering and faster results");
  }
  if (UI.usemathjax){
    var script = document.createElement("script");
    script.type = "text/javascript";
    script.src  = "http://cdn.mathjax.org/mathjax/latest/MathJax.js?config=TeX-AMS-MML_HTMLorMML";
    document.getElementsByTagName("head")[0].appendChild(script);
  }
  giaceval=Module.cwrap('caseval',  'string', ['string']);
  giac3d = Module.cwrap('_ZN4giac13giac_rendererEPKc','number', ['string']);
  giaceval('set_language(1);');
  giaceval('factor(x^4-1)');
  giaceval('sin(x+y)+f(t)');
 // if (confirm('Exec commands?')) UI.exec(document.documentElement);
 };
</script>
}
\else
The HTML version of this text has interactive fields running Giac
computer algebra system commands.
\fi
}
\newcommand{\loadgiacjsonline}{
\ifhevea
\@print{
<button onclick="UI.exec(document.documentElement);" title='Click here to execute all commands'>Execute</button>
<canvas id='canvas' width=0 height=0   onmousedown="UI.canvas_pushed=true;UI.canvas_lastx=event.clientX; UI.canvas_lasty=event.clientY;"  onmouseup="UI.canvas_pushed=false;" onmousemove="UI.canvas_mousemove(event,'')"></canvas>
<textarea id="output" rows=2 cols="80" title="Console"></textarea>
<script language="javascript"> 
var Module = { preRun: [], postRun: [], 
        print: (function() {
          var element = document.getElementById('output');
          element.value = ''; // clear browser cache
          return function(text) {
            element.style.display='inherit';
            element.value += text + "\n";
            element.scrollTop = 99999; // focus on bottom
          };
        })(),
      canvas: document.getElementById('canvas'),};
</script>
<script src="http://www-fourier.ujf-grenoble.fr/~parisse/giac.js" async></script> 
<script language="javascript">
 var UI = {
  histcount:0,
  usemathjax:false,
  canvas_pushed:false,
  canvas_lastx:0,
  canvas_lasty:0,
  canvas_mousemove:function(event,no){
    if (UI.canvas_pushed){
      // Module.print(event.clientX);
      if (UI.canvas_lastx!=event.clientX){
        if (event.clientX>UI.canvas_lastx)
          giac3d('r'+no);
        else
          giac3d('l'+no);
        UI.canvas_lastx=event.clientX;
      }
      if (UI.canvas_lasty!=event.clientY){
        if (event.clientY>UI.canvas_lasty)
          giac3d('d'+no);
        else
          giac3d('u'+no);
        UI.canvas_lasty=event.clientY;
      }
    }
  },  
  caseval:function(text){
    var s="not evaled",err;
    try {
       s= giaceval(text);
    } catch (err) { s=err.message;}
    var is_3d=s.length>5 && s.substr(0,5)=='gl3d ';
    if (is_3d){
	var n3d=s.substr(5,s.length-5);
	s = '<canvas id="gl3d_'+n3d+'" onmousedown="UI.canvas_pushed=true;UI.canvas_lastx=event.clientX; UI.canvas_lasty=event.clientY;" onmouseup="UI.canvas_pushed=false;" onmousemove="UI.canvas_mousemove(event,'+n3d+')" width=400 height=250></canvas>';
    }
   // console.log(s);
    return s;
  },
  eval_form: function(field){
    giaceval('assume('+field.name.value+'='+field.valname.value+')');
    var s=UI.caseval(field.prog.value);
    var is_svg=s.substr(1,4)=='<svg';
    if (is_svg) field.parentNode.lastChild.innerHTML=s.substr(1,s.length-2);
    else field.parentNode.lastChild.innerHTML=s;
   UI.render_canvas(field.parentNode.lastChild);
  },
  latexeval:function(text){
    var tmp=text;
     if (UI.usemathjax){
       tmp=giaceval('latex(quote('+tmp+'))');
       tmp='$$'+tmp.substr(1,tmp.length-2)+'$$';
       return tmp;
     }
     tmp=giaceval('mathml(quote('+tmp+',1))');
     tmp=tmp.substr(1,tmp.length-2);
    return tmp;   
  },
  ckenter:function(event,field,mode){
    var key = event.keyCode;
    if (key != 13 || event.shiftKey) return true;
    var tmp=field.value;
    tmp=UI.caseval(tmp);
    if (mode==1){
      if (tmp.charCodeAt(0)==34) tmp=tmp.substr(1,tmp.length-2); 
   }
   if (mode==2){
     tmp=UI.latexeval(tmp);
   }
   field.nextSibling.nextSibling.innerHTML=tmp;
   UI.render_canvas(field.nextSibling.nextSibling);
   if (UI.usemathjax) MathJax.Hub.Queue(["Typeset",MathJax.Hub,field.nextSibling.nextSibling]);
   if (event.preventDefault) event.preventDefault();
    return false;
  },
  exec: function(field){
     if (field.nodeName=="BUTTON"){
        field.click();
        return;
     }
     var f=field.firstChild;
     while (f){
       UI.exec(f);
       f=f.nextSibling;
     }
   }
 };
 window.onload = function(e){
   var isFirefox = typeof InstallTrigger !== 'undefined';   // Firefox 1.0+
   var isSafari = Object.prototype.toString.call(window.HTMLElement).indexOf('Constructor') > 0;
  var ua = window.navigator.userAgent;
  var old_ie = ua.indexOf('MSIE ');
  var new_ie = ua.indexOf('Trident/');
  if ((!isFirefox && !isSafari) || (old_ie > -1) || (new_ie > -1) || window.chrome){
     UI.usemathjax=true;
     alert("Your browser does not support MathML, using MathJax for 2d rendering. Consider switching to Firefox for better rendering and faster results");
  }
  if (UI.usemathjax){
    var script = document.createElement("script");
    script.type = "text/javascript";
    script.src  = "http://cdn.mathjax.org/mathjax/latest/MathJax.js?config=TeX-AMS-MML_HTMLorMML";
    document.getElementsByTagName("head")[0].appendChild(script);
  }
  giaceval=Module.cwrap('caseval',  'string', ['string']);
  giac3d = Module.cwrap('_ZN4giac13giac_rendererEPKc','number', ['string']);
  giaceval('set_language(1);');
  giaceval('factor(x^4-1)');
  giaceval('sin(x+y)+f(t)');
 // if (confirm('Exec commands?')) UI.exec(document.documentElement);
 };
</script>
}
\else
The HTML version of this text has interactive fields running Giac
computer algebra system commands.
\fi
}
\ifhevea
\newenvironment{giacprog}{
\verbatim}
{\endverbatim 
\@print{<button onclick="var field=parentNode.previousSibling; var tmp=field.innerHTML;if(tmp.length==0) tmp=field.value;var t=createElement('TEXTAREA');t.style.fontSize=16;t.cols=60;t.rows=10;t.value=tmp;tmp=UI.caseval(tmp);if (tmp.charCodeAt(0)==34) tmp=tmp.substr(1,tmp.length-2);nextSibling.innerHTML=tmp; UI.render_canvas(nextSibling.innerHTML); field.parentNode.insertBefore(t,field);field.parentNode.removeChild(field);">ok</button><span></span><br>
}
}
\else
\newenvironment{giacprog}
{
\VerbatimEnvironment
\begin{Verbatim}
}
{
\end{Verbatim}
}
\fi

% for example \giaccmd{factor}{x^4-1}
\newcommand{\giaccmd}[3][style="width:400px;height:20px;font-size:large"]{
\ifhevea
\@print{<textarea}
\@getprint{#1>#3}
\@print{</textarea><button onclick="var tmp=UI.caseval(}
\@getprint{'#2('}
\@print{+previousSibling.value+')');if (tmp.charCodeAt(0)==34) tmp=tmp.substr(1,tmp.length-2); nextSibling.innerHTML='&nbsp;'+tmp;UI.render_canvas(nextSibling)">}
\@getprint{#2}
\@print{</button><span></span><br>}
\else
\lstinline@#2(#3)@
\fi
}
\newcommand{\giacinput}[2][style="width:400px;height:20px;font-size:large"]{
\ifhevea
\@print{<textarea onkeypress="UI.ckenter(event,this,1)" }
\@getprint{#1>#2}
\@print{</textarea><button onclick="var tmp=UI.caseval(previousSibling.value);if (tmp.charCodeAt(0)==34) tmp=tmp.substr(1,tmp.length-2); nextSibling.innerHTML='&nbsp;'+tmp;UI.render_canvas(nextSibling);">ok</button><span></span><br>}
\else
\lstinline@#2@
\fi
}
\newcommand{\giaccmdmath}[3][style="width:400px;height:20px;font-size:large"]{\ifhevea
\begin{rawhtml}<br><textarea \end{rawhtml}
\@getprint{#1>#3}
\begin{rawhtml}</textarea><button onclick="var tmp=UI.caseval(\end{rawhtml} 
\@getprint{'#2('+}
\begin{rawhtml}previousSibling.value+')');  tmp=UI.latexeval(tmp);nextSibling.innerHTML='&nbsp;'+tmp; if (UI.usemathjax) MathJax.Hub.Queue(['Typeset',MathJax.Hub,nextSibling]);
">\end{rawhtml}
\@getprint{#2}
\begin{rawhtml}</button><span></span><br>\end{rawhtml}
\else
\lstinline@#2(#3)@
\fi
}
\newcommand{\giacinputmath}[2][style="width:400px;height:20px;font-size:large"]{\ifhevea
\begin{rawhtml}<br><textarea onkeypress="UI.ckenter(event,this,2)" \end{rawhtml}
\@getprint{#1>#2} 
\begin{rawhtml}</textarea><button onclick="var tmp=UI.caseval(previousSibling.value); tmp=UI.latexeval(tmp);nextSibling.innerHTML='&nbsp;'+tmp; if (UI.usemathjax) MathJax.Hub.Queue(['Typeset',MathJax.Hub,nextSibling])">ok</button><span></span><br>\end{rawhtml}
\else
\lstinline@#2@
\fi
}
\newcommand{\giaccmdbigmath}[3][style="width:800px;height:20px;font-size:large"]{\ifhevea
\begin{rawhtml}<br><textarea \end{rawhtml}
\@getprint{#1>#3} 
\begin{rawhtml}</textarea><button onclick="var tmp=UI.caseval(\end{rawhtml} 
\@getprint{'#2('+}
\begin{rawhtml}previousSibling.value+')');  tmp=UI.latexeval(tmp);nextSibling.innerHTML=tmp; if (UI.usemathjax) MathJax.Hub.Queue(['Typeset',MathJax.Hub,nextSibling])">\end{rawhtml}
\@getprint{#2}
\begin{rawhtml}</button><div style="width:800px;max-height:200px;overflow:auto;color:blue;text-align:center"></div><br>\end{rawhtml}
\else
\lstinline@#2(#3)@
\fi
}
\newcommand{\giacinputbigmath}[2][style="width:800px;height:20px;font-size:large"]{\ifhevea
\begin{rawhtml}<div><textarea onkeypress="UI.ckenter(event,this,2)" \end{rawhtml}
\@getprint{#1>#2} 
\begin{rawhtml}</textarea><button onclick="var tmp=UI.caseval(previousSibling.value); tmp=UI.latexeval(tmp);nextSibling.innerHTML=tmp; if (UI.usemathjax) MathJax.Hub.Queue(['Typeset',MathJax.Hub,nextSibling])">ok</button><div style="width:800px;max-height:200px;overflow:auto;color:blue;text-align:center"></div></div>\end{rawhtml}
\else
\lstinline@#2@
\fi
}
\newcommand{\giaclink}[2][Test online]{\ifhevea
\begin{rawhtml}<a href=\end{rawhtml}\@getprint{"#2"}
\begin{rawhtml} target="_blank">\end{rawhtml}
\@getprint{#1}
\begin{rawhtml}</a>\end{rawhtml}
\else
\fi
}
% \giacslider{name}{mini}{maxi}{step}{value}{prog}
\newcommand{\giacslider}[6]{
\ifhevea
\begin{rawhtml}
<div><form onsubmit="setTimeout(function(){UI.eval_form(form);});return false;">
<input type="text" name="name" size="1" value=
\end{rawhtml}
\@getprint{"#1">}
\begin{rawhtml}
=<input type="number" name="valname" onchange="UI.eval_form(form);" value=
\end{rawhtml}
\@getprint{"#5">}
\begin{rawhtml}
<input type="button" value="-" onclick="valname.value -= stepname.value;UI.eval_form(form);">
<input type="button" value="+" onclick="valname.value -= -stepname.value;UI.eval_form(form);">
<input type="number" name="stepname" value=
\end{rawhtml}
\@getprint{"#4">}
\begin{rawhtml}
<input type="range" name="rangename"
onclick="valname.value=value;UI.eval_form(form);" value=
\end{rawhtml}
\@getprint{"#5" min="#2" max="#3" step="#4">}
\begin{rawhtml}
<textarea name="prog" onchange="UI.eval_form(form)" style="width:400px;height:120px;vertical-align:bottom;font-size:large">
\end{rawhtml}
\@getprint{#6}
\begin{rawhtml}
</textarea>
</form>
<span>Not evaled</span></div>
\end{rawhtml}
\else
\fi
}
%%%%%%%%%%%%%%%%%%%%%%%%%%%%%%%
%% giac/hevea end code definitions
%%%%%%%%%%%%%%%%%%%%%%%%%%%%%%%



\newtheorem{thm}{Theorem}
\newtheorem{defn}[thm]{Definition}
\newtheorem{prop}[thm]{Proposition}
\newtheorem{lemma}[thm]{Lemma}
\newtheorem{example}[thm]{Example}


\newcommand{\R}{{\mathbb{R}}}
\newcommand{\C}{{\mathbb{C}}}
\newcommand{\Z}{{\mathbb{Z}}}
\newcommand{\N}{{\mathbb{N}}}
\newcommand{\Q}{{\mathbb{Q}}}
\newcommand{\tr}{\mbox{tr\,}}


\title {Examples of interactive computations inside a \LaTeX\ file
  compiled to HTML.}
\author{B. Parisse\\Institut Fourier\\UMR 5582 du CNRS
\\Université de Grenoble}

\date{2015}

\makeindex

\begin{document}
%\loadgiacjsonline
%\loadgiacjs
\begin{giacjsonline}
\maketitle

\tableofcontents

\printindex

\section{Description}
This \LaTeX\ source will output an interactive HTML file if you compile
with {\tt hevea} (tested with \verb|hevea 2.23|), where interactive
computations are done with the computer algebra system Giac.

\section{Install}
Please install
\ahref{http://hevea.inria.fr}{hevea}.
Copy \ahref{http://www-fourier.ujf-grenoble.fr/\%7eparisse/giac/giac.tex}{giac.tex}
and \ahref{http://hevea.inria.fr/distri/hevea.sty}{hevea.sty}
in the same folder as your source file.
You can also get a copy of the source
\ahref{http://www-fourier.ujf-grenoble.fr/\%7eparisse/giac/test.tex}{test.tex} 
for this file and check your installation by compiling it\\
\verb|hevea test|

\section{Commands}
You must first enter the command \verb|%%%%%%%%%%%%%%%%%%%%%%%%%%%%%%%
%% giac/hevea interaction code definitions
%%%%%%%%%%%%%%%%%%%%%%%%%%%%%%%
%% Type one of the commands \loadgiacjs or \loadgiacjsonline 
%% to load the javascript code (from hard disk or Internet)
%% Commands \giacinput{} or \giacinputmath{} or \giacinputbigmath{} 
%% \giaccmd{} or \giaccmdmath{} or \giaccmdbigmath{} 
%% or \begin{giacprog}...\end{giacprog}
%% contact hevea to output <frameset> </frameset> for canvas and menu?
\usepackage{hevea} 
\usepackage{listings}
\usepackage{fancyvrb}
\newcommand{\loadgiacjs}{
\ifhevea
\@print{
<button onclick="UI.exec(document.documentElement);" title='Click here to execute all commands'>Execute</button>
<canvas id='canvas' width=0 height=0   onmousedown="UI.canvas_pushed=true;UI.canvas_lastx=event.clientX; UI.canvas_lasty=event.clientY;"  onmouseup="UI.canvas_pushed=false;" onmousemove="UI.canvas_mousemove(event,'')"></canvas>
<textarea id="output" rows=2 cols="80" title="Console"></textarea>
<script language="javascript"> 
var Module = { preRun: [], postRun: [], 
        print: (function() {
          var element = document.getElementById('output');
          element.value = ''; // clear browser cache
          return function(text) {
            element.style.display='inherit';
            element.value += text + "\n";
            element.scrollTop = 99999; // focus on bottom
          };
        })(),
      canvas: document.getElementById('canvas'),};
</script>
<script src="file:///usr/share/giac/doc/giac.js" async></script> 
<script language="javascript"> 
 var UI = {
  histcount:0,
  usemathjax:false,
  canvas_pushed:false,
  canvas_lastx:0,
  canvas_lasty:0,
  canvas_mousemove:function(event,no){
    if (UI.canvas_pushed){
      // Module.print(event.clientX);
      if (UI.canvas_lastx!=event.clientX){
        if (event.clientX>UI.canvas_lastx)
          giac3d('r'+no);
        else
          giac3d('l'+no);
        UI.canvas_lastx=event.clientX;
      }
      if (UI.canvas_lasty!=event.clientY){
        if (event.clientY>UI.canvas_lasty)
          giac3d('d'+no);
        else
          giac3d('u'+no);
        UI.canvas_lasty=event.clientY;
      }
    }
  },  
  render_canvas:function(field){
   var n=field.id;
   if (n && n.length>5 && n.substr(0,5)=='gl3d_'){
    Module.print(n);
    var n3d=n.substr(5,n.length-5);
    giac3d(n3d);
    return;
   }
   var f=field.firstChild;
   for (;f;f=f.nextSibling){
     UI.render_canvas(f);
   }
  },
  caseval:function(text){
    var s="not evaled",err;
    try {
       s= giaceval(text);
    } catch (err) { s=err.message;}
    var is_3d=s.length>5 && s.substr(0,5)=='gl3d ';
    if (is_3d){
	var n3d=s.substr(5,s.length-5);
	s = '<canvas id="gl3d_'+n3d+'" onmousedown="UI.canvas_pushed=true;UI.canvas_lastx=event.clientX; UI.canvas_lasty=event.clientY;" onmouseup="UI.canvas_pushed=false;" onmousemove="UI.canvas_mousemove(event,'+n3d+')" width=400 height=250></canvas>';
    }
   //console.log(s);
    return s;
  },
  eval_form: function(field){
    giaceval('assume('+field.name.value+'='+field.valname.value+')');
    var s=UI.caseval(field.prog.value);
    var is_svg=s.substr(1,4)=='<svg';
    if (is_svg) field.parentNode.lastChild.innerHTML=s.substr(1,s.length-2);
    else field.parentNode.lastChild.innerHTML=s;
   UI.render_canvas(field.parentNode.lastChild);
  },
  latexeval:function(text){
    var tmp=text;
     if (UI.usemathjax){
       tmp=giaceval('latex(quote('+tmp+'))');
       tmp='$$'+tmp.substr(1,tmp.length-2)+'$$';
       return tmp;
     }
     tmp=giaceval('mathml(quote('+text+',1))');
     tmp=tmp.substr(1,tmp.length-2);
    return tmp;   
  },
  ckenter:function(event,field,mode){
    var key = event.keyCode;
    if (key != 13 || event.shiftKey) return true;
    var tmp=field.value;
    tmp=UI.caseval(tmp);
    if (mode==1){
      if (tmp.charCodeAt(0)==34) tmp=tmp.substr(1,tmp.length-2); 
   }
   if (mode==2){
     tmp=UI.latexeval(text);
   }
   field.nextSibling.nextSibling.innerHTML='&nbsp;'+tmp;
   UI.render_canvas(field.nextSibling.nextSibling);
   if (UI.usemathjax) MathJax.Hub.Queue(["Typeset",MathJax.Hub,field.nextSibling.nextSibling]);
   if (event.preventDefault) event.preventDefault();
    return false;
  },
   exec: function(field){
     if (field.nodeName=="BUTTON"){
        field.click();
        return;
     }
     if (field.nodeName=="FORM"){
        UI.eval_form(field);
        return;
     }
     var f=field.firstChild;
     while (f){
       UI.exec(f);
       f=f.nextSibling;
     }
   }
 };
 window.onload = function(e){
   var isFirefox = typeof InstallTrigger !== 'undefined';   // Firefox 1.0+
   var isSafari = Object.prototype.toString.call(window.HTMLElement).indexOf('Constructor') > 0;
  var ua = window.navigator.userAgent;
  var old_ie = ua.indexOf('MSIE ');
  var new_ie = ua.indexOf('Trident/');
  if ((!isFirefox && !isSafari) || (old_ie > -1) || (new_ie > -1) || window.chrome){
     UI.usemathjax=true;
     alert("Your browser does not support MathML, using MathJax for 2d rendering. Consider switching to Firefox for better rendering and faster results");
  }
  if (UI.usemathjax){
    var script = document.createElement("script");
    script.type = "text/javascript";
    script.src  = "http://cdn.mathjax.org/mathjax/latest/MathJax.js?config=TeX-AMS-MML_HTMLorMML";
    document.getElementsByTagName("head")[0].appendChild(script);
  }
  giaceval=Module.cwrap('caseval',  'string', ['string']);
  giac3d = Module.cwrap('_ZN4giac13giac_rendererEPKc','number', ['string']);
  giaceval('set_language(1);');
  giaceval('factor(x^4-1)');
  giaceval('sin(x+y)+f(t)');
 // if (confirm('Exec commands?')) UI.exec(document.documentElement);
 };
</script>
}
\else
The HTML version of this text has interactive fields running Giac
computer algebra system commands.
\fi
}
\newcommand{\loadgiacjsonline}{
\ifhevea
\@print{
<button onclick="UI.exec(document.documentElement);" title='Click here to execute all commands'>Execute</button>
<canvas id='canvas' width=0 height=0   onmousedown="UI.canvas_pushed=true;UI.canvas_lastx=event.clientX; UI.canvas_lasty=event.clientY;"  onmouseup="UI.canvas_pushed=false;" onmousemove="UI.canvas_mousemove(event,'')"></canvas>
<textarea id="output" rows=2 cols="80" title="Console"></textarea>
<script language="javascript"> 
var Module = { preRun: [], postRun: [], 
        print: (function() {
          var element = document.getElementById('output');
          element.value = ''; // clear browser cache
          return function(text) {
            element.style.display='inherit';
            element.value += text + "\n";
            element.scrollTop = 99999; // focus on bottom
          };
        })(),
      canvas: document.getElementById('canvas'),};
</script>
<script src="http://www-fourier.ujf-grenoble.fr/~parisse/giac.js" async></script> 
<script language="javascript">
 var UI = {
  histcount:0,
  usemathjax:false,
  canvas_pushed:false,
  canvas_lastx:0,
  canvas_lasty:0,
  canvas_mousemove:function(event,no){
    if (UI.canvas_pushed){
      // Module.print(event.clientX);
      if (UI.canvas_lastx!=event.clientX){
        if (event.clientX>UI.canvas_lastx)
          giac3d('r'+no);
        else
          giac3d('l'+no);
        UI.canvas_lastx=event.clientX;
      }
      if (UI.canvas_lasty!=event.clientY){
        if (event.clientY>UI.canvas_lasty)
          giac3d('d'+no);
        else
          giac3d('u'+no);
        UI.canvas_lasty=event.clientY;
      }
    }
  },  
  caseval:function(text){
    var s="not evaled",err;
    try {
       s= giaceval(text);
    } catch (err) { s=err.message;}
    var is_3d=s.length>5 && s.substr(0,5)=='gl3d ';
    if (is_3d){
	var n3d=s.substr(5,s.length-5);
	s = '<canvas id="gl3d_'+n3d+'" onmousedown="UI.canvas_pushed=true;UI.canvas_lastx=event.clientX; UI.canvas_lasty=event.clientY;" onmouseup="UI.canvas_pushed=false;" onmousemove="UI.canvas_mousemove(event,'+n3d+')" width=400 height=250></canvas>';
    }
   // console.log(s);
    return s;
  },
  eval_form: function(field){
    giaceval('assume('+field.name.value+'='+field.valname.value+')');
    var s=UI.caseval(field.prog.value);
    var is_svg=s.substr(1,4)=='<svg';
    if (is_svg) field.parentNode.lastChild.innerHTML=s.substr(1,s.length-2);
    else field.parentNode.lastChild.innerHTML=s;
   UI.render_canvas(field.parentNode.lastChild);
  },
  latexeval:function(text){
    var tmp=text;
     if (UI.usemathjax){
       tmp=giaceval('latex(quote('+tmp+'))');
       tmp='$$'+tmp.substr(1,tmp.length-2)+'$$';
       return tmp;
     }
     tmp=giaceval('mathml(quote('+tmp+',1))');
     tmp=tmp.substr(1,tmp.length-2);
    return tmp;   
  },
  ckenter:function(event,field,mode){
    var key = event.keyCode;
    if (key != 13 || event.shiftKey) return true;
    var tmp=field.value;
    tmp=UI.caseval(tmp);
    if (mode==1){
      if (tmp.charCodeAt(0)==34) tmp=tmp.substr(1,tmp.length-2); 
   }
   if (mode==2){
     tmp=UI.latexeval(tmp);
   }
   field.nextSibling.nextSibling.innerHTML=tmp;
   UI.render_canvas(field.nextSibling.nextSibling);
   if (UI.usemathjax) MathJax.Hub.Queue(["Typeset",MathJax.Hub,field.nextSibling.nextSibling]);
   if (event.preventDefault) event.preventDefault();
    return false;
  },
  exec: function(field){
     if (field.nodeName=="BUTTON"){
        field.click();
        return;
     }
     var f=field.firstChild;
     while (f){
       UI.exec(f);
       f=f.nextSibling;
     }
   }
 };
 window.onload = function(e){
   var isFirefox = typeof InstallTrigger !== 'undefined';   // Firefox 1.0+
   var isSafari = Object.prototype.toString.call(window.HTMLElement).indexOf('Constructor') > 0;
  var ua = window.navigator.userAgent;
  var old_ie = ua.indexOf('MSIE ');
  var new_ie = ua.indexOf('Trident/');
  if ((!isFirefox && !isSafari) || (old_ie > -1) || (new_ie > -1) || window.chrome){
     UI.usemathjax=true;
     alert("Your browser does not support MathML, using MathJax for 2d rendering. Consider switching to Firefox for better rendering and faster results");
  }
  if (UI.usemathjax){
    var script = document.createElement("script");
    script.type = "text/javascript";
    script.src  = "http://cdn.mathjax.org/mathjax/latest/MathJax.js?config=TeX-AMS-MML_HTMLorMML";
    document.getElementsByTagName("head")[0].appendChild(script);
  }
  giaceval=Module.cwrap('caseval',  'string', ['string']);
  giac3d = Module.cwrap('_ZN4giac13giac_rendererEPKc','number', ['string']);
  giaceval('set_language(1);');
  giaceval('factor(x^4-1)');
  giaceval('sin(x+y)+f(t)');
 // if (confirm('Exec commands?')) UI.exec(document.documentElement);
 };
</script>
}
\else
The HTML version of this text has interactive fields running Giac
computer algebra system commands.
\fi
}
\ifhevea
\newenvironment{giacprog}{
\verbatim}
{\endverbatim 
\@print{<button onclick="var field=parentNode.previousSibling; var tmp=field.innerHTML;if(tmp.length==0) tmp=field.value;var t=createElement('TEXTAREA');t.style.fontSize=16;t.cols=60;t.rows=10;t.value=tmp;tmp=UI.caseval(tmp);if (tmp.charCodeAt(0)==34) tmp=tmp.substr(1,tmp.length-2);nextSibling.innerHTML=tmp; UI.render_canvas(nextSibling.innerHTML); field.parentNode.insertBefore(t,field);field.parentNode.removeChild(field);">ok</button><span></span><br>
}
}
\else
\newenvironment{giacprog}
{
\VerbatimEnvironment
\begin{Verbatim}
}
{
\end{Verbatim}
}
\fi

% for example \giaccmd{factor}{x^4-1}
\newcommand{\giaccmd}[3][style="width:400px;height:20px;font-size:large"]{
\ifhevea
\@print{<textarea}
\@getprint{#1>#3}
\@print{</textarea><button onclick="var tmp=UI.caseval(}
\@getprint{'#2('}
\@print{+previousSibling.value+')');if (tmp.charCodeAt(0)==34) tmp=tmp.substr(1,tmp.length-2); nextSibling.innerHTML='&nbsp;'+tmp;UI.render_canvas(nextSibling)">}
\@getprint{#2}
\@print{</button><span></span><br>}
\else
\lstinline@#2(#3)@
\fi
}
\newcommand{\giacinput}[2][style="width:400px;height:20px;font-size:large"]{
\ifhevea
\@print{<textarea onkeypress="UI.ckenter(event,this,1)" }
\@getprint{#1>#2}
\@print{</textarea><button onclick="var tmp=UI.caseval(previousSibling.value);if (tmp.charCodeAt(0)==34) tmp=tmp.substr(1,tmp.length-2); nextSibling.innerHTML='&nbsp;'+tmp;UI.render_canvas(nextSibling);">ok</button><span></span><br>}
\else
\lstinline@#2@
\fi
}
\newcommand{\giaccmdmath}[3][style="width:400px;height:20px;font-size:large"]{\ifhevea
\begin{rawhtml}<br><textarea \end{rawhtml}
\@getprint{#1>#3}
\begin{rawhtml}</textarea><button onclick="var tmp=UI.caseval(\end{rawhtml} 
\@getprint{'#2('+}
\begin{rawhtml}previousSibling.value+')');  tmp=UI.latexeval(tmp);nextSibling.innerHTML='&nbsp;'+tmp; if (UI.usemathjax) MathJax.Hub.Queue(['Typeset',MathJax.Hub,nextSibling]);
">\end{rawhtml}
\@getprint{#2}
\begin{rawhtml}</button><span></span><br>\end{rawhtml}
\else
\lstinline@#2(#3)@
\fi
}
\newcommand{\giacinputmath}[2][style="width:400px;height:20px;font-size:large"]{\ifhevea
\begin{rawhtml}<br><textarea onkeypress="UI.ckenter(event,this,2)" \end{rawhtml}
\@getprint{#1>#2} 
\begin{rawhtml}</textarea><button onclick="var tmp=UI.caseval(previousSibling.value); tmp=UI.latexeval(tmp);nextSibling.innerHTML='&nbsp;'+tmp; if (UI.usemathjax) MathJax.Hub.Queue(['Typeset',MathJax.Hub,nextSibling])">ok</button><span></span><br>\end{rawhtml}
\else
\lstinline@#2@
\fi
}
\newcommand{\giaccmdbigmath}[3][style="width:800px;height:20px;font-size:large"]{\ifhevea
\begin{rawhtml}<br><textarea \end{rawhtml}
\@getprint{#1>#3} 
\begin{rawhtml}</textarea><button onclick="var tmp=UI.caseval(\end{rawhtml} 
\@getprint{'#2('+}
\begin{rawhtml}previousSibling.value+')');  tmp=UI.latexeval(tmp);nextSibling.innerHTML=tmp; if (UI.usemathjax) MathJax.Hub.Queue(['Typeset',MathJax.Hub,nextSibling])">\end{rawhtml}
\@getprint{#2}
\begin{rawhtml}</button><div style="width:800px;max-height:200px;overflow:auto;color:blue;text-align:center"></div><br>\end{rawhtml}
\else
\lstinline@#2(#3)@
\fi
}
\newcommand{\giacinputbigmath}[2][style="width:800px;height:20px;font-size:large"]{\ifhevea
\begin{rawhtml}<div><textarea onkeypress="UI.ckenter(event,this,2)" \end{rawhtml}
\@getprint{#1>#2} 
\begin{rawhtml}</textarea><button onclick="var tmp=UI.caseval(previousSibling.value); tmp=UI.latexeval(tmp);nextSibling.innerHTML=tmp; if (UI.usemathjax) MathJax.Hub.Queue(['Typeset',MathJax.Hub,nextSibling])">ok</button><div style="width:800px;max-height:200px;overflow:auto;color:blue;text-align:center"></div></div>\end{rawhtml}
\else
\lstinline@#2@
\fi
}
\newcommand{\giaclink}[2][Test online]{\ifhevea
\begin{rawhtml}<a href=\end{rawhtml}\@getprint{"#2"}
\begin{rawhtml} target="_blank">\end{rawhtml}
\@getprint{#1}
\begin{rawhtml}</a>\end{rawhtml}
\else
\fi
}
% \giacslider{name}{mini}{maxi}{step}{value}{prog}
\newcommand{\giacslider}[6]{
\ifhevea
\begin{rawhtml}
<div><form onsubmit="setTimeout(function(){UI.eval_form(form);});return false;">
<input type="text" name="name" size="1" value=
\end{rawhtml}
\@getprint{"#1">}
\begin{rawhtml}
=<input type="number" name="valname" onchange="UI.eval_form(form);" value=
\end{rawhtml}
\@getprint{"#5">}
\begin{rawhtml}
<input type="button" value="-" onclick="valname.value -= stepname.value;UI.eval_form(form);">
<input type="button" value="+" onclick="valname.value -= -stepname.value;UI.eval_form(form);">
<input type="number" name="stepname" value=
\end{rawhtml}
\@getprint{"#4">}
\begin{rawhtml}
<input type="range" name="rangename"
onclick="valname.value=value;UI.eval_form(form);" value=
\end{rawhtml}
\@getprint{"#5" min="#2" max="#3" step="#4">}
\begin{rawhtml}
<textarea name="prog" onchange="UI.eval_form(form)" style="width:400px;height:120px;vertical-align:bottom;font-size:large">
\end{rawhtml}
\@getprint{#6}
\begin{rawhtml}
</textarea>
</form>
<span>Not evaled</span></div>
\end{rawhtml}
\else
\fi
}
%%%%%%%%%%%%%%%%%%%%%%%%%%%%%%%
%% giac/hevea end code definitions
%%%%%%%%%%%%%%%%%%%%%%%%%%%%%%%
| in the
preamble of your \LaTeX\ source file
and add one of the commands \verb|\begin{giacjs}|\index{giacjs} or 
\verb|\begin{giacjsonline}|\index{giacjsonline}
just after \verb|\begin{document}|: the difference is that the javascript
kernel \verb|giac.js|
will either be found on the hard disk (assuming that Giac/Xcas is
installed on the target computer) or downloaded from Internet.
You must also add the command \verb|\end{giacjs}| or
\verb|\end{giacjsonline}| just before \verb|\end{document}|.
You should add the command \verb|\tableofcontents|\index{table} then
\verb|\printindex|\index{index} just after
\verb|\begin{giacjsonline}| (and add \verb|\makeindex| somewhere).

Inline command with Mathml or 2d graph output 
\verb|\giacinputmath{}| or \verb|\giaccmdmath{}{}|~:\\
\verb|\giacinputmath{factor(x^10-1)}|\index{giacinputmath}\\
\giacinputmath{factor(x^10-1)}\\
With an optional style argument\\
\verb|\giacinputmath[style="width:200px;height:20px;font-size:large"]{factor(x^10-1)}|\\
\giacinputmath[style="width:200px;height:20px;font-size:large"]{factor(x^10-1)}\\
\verb|\giaccmdmath{factor}{x^4-1}|\index{giaccmdmath}\\
\giaccmdmath{factor}{x^4-1}\\
\verb|\giaccmdmath[style="width:200px;height:20px;font-size:large"]{factor}{x^4-1}|\\
\giaccmdmath[style="width:200px;height:20px;font-size:large"]{factor}{x^4-1}

Outline command with mathml output
\verb|\giacinputbigmath{}| or \verb|\giaccmdbigmath{}{}|~:\\
\verb|\giacinputbigmath{factor(x^100-1)}|\index{giacinputbigmath}\\
\giacinputbigmath{factor(x^100-1)}\\
\verb|\giacinputbigmath[style="width:600px;height:20px;font-size:large"]{factor(x^100-1)}|\\
\giacinputbigmath[style="width:600px;height:20px;font-size:large"]{factor(x^100-1)}\\
\verb|\giaccmdbigmath{factor}{x^100-1}|\index{giaccmdbigmath}\\
\giaccmdbigmath{factor}{x^100-1}\\
\verb|\giaccmdbigmath[style="width:600px;height:20px;font-size:large"]{factor}{x^100-1}|\\
\giaccmdbigmath[style="width:600px;height:20px;font-size:large"]{factor}{x^100-1}

Inline command example with text or plot output 
\verb|\giacinput| and \verb|\giacinputbig|\index{giacinputbig}, 
example:\\
\verb|\giacinput{factor(x^4-1)}|\index{giacinput}~:\\
\giacinput{factor(x^4-1)}\\
Same command with optional style argument\\
\verb|\giacinput[style="width:200px;height:20px;font-size:large"]{plot(sin(x))}|\\
\giacinput[style="width:200px;height:20px;font-size:large"]{plot(sin(x))}

A button with a command applied on the field entry with
\verb|\giaccmd|, example\\
\verb|\giaccmd{factor}{x^4-1}|\index{giaccmd}~:\\
\giaccmd{factor}{x^4-1}\\
with optional style argument\\
\verb|\giaccmd[style="width:200px;height:20px;font-size:large"]{factor}{x^4-1}|\\
\giaccmd[style="width:200px;height:20px;font-size:large"]{factor}{x^4-1}

For a program\index{giacprog} or multi-line commands\\
\verb|\begin{giacprog}...\end{giacprog}|, example homemade absolute value\\
\begin{giacprog}
f(x):={
  local y;
  if x<0 then y:=-x; else y=x; fi;
  return y;
}
\end{giacprog}

A link\index{giaclink} to Xcas offline with a few commands\\
\verb|\giaclink{http://www-fourier.ujf-grenoble.fr/\%7eparisse/xcasen.html#+factor(x^4-1)&+a:=idn(3)&}|\\
\giaclink{http://www-fourier.ujf-grenoble.fr/\%7eparisse/xcasen.html#+factor(x^4-1)&+a:=idn(3)&}

A slider\\ 
\verb|\giacslider{a}{-5}{5}{0.1}{0}{plot(sin(a*x))}|\index{giacslider}\\
\giacslider{a}{-5}{5}{0.1}{0}{plot(sin(a*x))}

\end{giacjsonline}
\end{document}
